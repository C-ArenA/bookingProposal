\section{Objetivos}

A continuación se cita el objetivo general propuesto por la FUDEP y se detallan después nuestros objetivos específicos desde un punto de vista técnico.

\subsection{Objetivo General}

Implementar un sistema informático institucional autoadministrable de seguimiento y monitoreo técnico articulado al presupuestario, 
en base a los objetivos, resultados, indicadores y metas de proyectos/programas sociales, que permita monitorear los avances cuantitativos y cualitativos de acuerdo al marco lógico u otras herramientas de gestión de proyectos/programas,
que genere información integrada, periódica, desglosada, sectorizada y otras variables de análisis de información, toma de decisiones oportunas, evaluaciones, etc.

\subsection{Objetivos Específicos}

\begin{itemize}
    \item Crear un módulo de acceso a usuarios para los distintos roles dentro de la institución
    \item Integrar el ingreso de usuarios con proveedores OAuth como ser Google, para facilitar la adopción de la herramienta
    \item Diseñar una estructura de datos que contemple las distintas variables requeridas por los planes y proyectos del FUDEP
    \item Proveer de una API que permita integrar el sistema con distintas plataformas digitales para difundir las actividades de la FUDEP
    \item Crear formularios sencillos e interactivos para el ingreso de datos
    \item Crear vistas específicas a cada proyecto y sus distintos componentes a partir de la metodología del Marco Lógico
    \item Proporcionar herramientas interactivas para llevar registro del estado de las distintas actividades de cada proyecto
    \item Sistematizar los procesos de asignación y solicitud de presupuestos así como los registros de gastos erogados en las distintas actividades
    \item Brindar reportes globales sobre el estado de todos los proyectos del sistema
    \item Otorgar información específica a cada entidad de datos existente
    \item Brindar un panel de administración amigable que contenga todo lo necesario para configurar el sistema
    \item Usar tecnologías vigentes, modernas, de código abierto y que cuenten con una gran comunidad de desarrolladores
    \item Implementar campos de auditoría en la base de datos del sistema, para garantizar su integridad
    \item Crear un plan de backups para garantizar que los datos sean correctamente conservados
    \item Brindar la posibilidad de acceder al sistema por internet en distintos territorios
    \item Integrar el lenguaje del sistema y la hermenéutica de su uso a las prácticas actuales del equipo contable de la FUDEP
    \item Recoger los requerimientos específicos de la Fundación La Paz para este sistema
    \item Promover el remplazo del papel por los medios digitales
    \item Brindar un sistema escalable y de fácil mantenimiento
    \item Aplicar el sistema a diferentes proyectos vigentes así como nuevas iniciativas.
    \item Brindar una interfaz de usuario (UI) y una experiencia de usuario (UX) profesional, amigable y estéticamente agradable
    \item Permitir exportar información a hojas de cálculo, gráficos, textos y otros de ofimática estándares.
    \item Contar con distintas estrategias de notificaciones para alarmas técnicas, como ser, mediante el uso del correo y otros canales de comunicación
    \item Capacitar al personal de la FUDEP durante la fase de pruebas e implementación del sistema.
\end{itemize}