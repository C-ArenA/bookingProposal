\section{Antecedentes}

La FUDEP (Fundación La Paz para el Desarrollo y la Participación) nace el año 1971 como Fundación San Gabriel consolidándose eventualmente como una institución de fomento y promoción de la participación de la población en base a los derechos humanos y el enfoque de género.

La participación de la población en distintos programas y proyectos son parte fundamental de su labor. Permitiendo beneficiar y convertir en actores relevantes a diversos participantes de la sociedad.  Incorporando a mujeres, niñas y niños en la toma de decisiones y la cogestión de proyectos.

Todo esto se refleja en su objetivo institucional que es, de acuerdo a lo expuesto en su sitio web:

\begin{displayquote}
Promover el diseño, ejecución y asesoramiento de programas, proyectos y acciones de promoción y gestión social tendientes a mejorar el nivel de vida de los destinatarios.
\end{displayquote}

Un objetivo que concuerda con su misión que expresa:

\begin{displayquote}
Somos una institución sin fines de lucro que impulsa/promueve el  empoderamiento, protagonismo y fortalezas de 
las niñas, niños, adolescentes, jóvenes y mujeres para asumir sus responsabilidades y exigir la restitución de sus derechos vulnerados.
\end{displayquote}

El pilar de los esfuerzos de la FUDEP, como podemos notar, es la gestión de programas y proyectos. Estos, al ser de tal importancia, conllevan a la aparición de ciertas necesidades, oportunidades y el claro interés de poder llevar a cabo las distintas actividades de forma transparente y que inspire confianza en todos los involucrados de tan importante misión.
