\section{Identificación de Problemas y Oportunidades}

% Propuesta debe ser enunciativa, no detallada
% Qué va a tener
% Parafrasear los términos de referencia

De acuerdo a lo expuesto por la Fundación La Paz para el Desarrollo y Participación en los TdR para la implementación de un Sistema Informático de Seguimiento y Monitoreo Institucional, el análisis realizado y expuesto en este documento y la experiencia técnica de 2IES, se identifican los siguientes problemas y oportunidades que fundamentan el desarrollo de este proyecto y que guiarán la consiguiente propuesta de solución y el plan de trabajo correspondiente:

\begin{itemize}
    \item Los Proyectos/Programas de la FUDEP requieren una manera de gestionarse acorde a los estándares modernos y mediante la utilización de las tecnologías de la información.
    \item No se cuenta con una plataforma digital que permita de manera práctica, sencilla y específica generar confianza en los financiadores, empleados, comunidad, socios y co partes.
    \item La transparencia institucional en la era digital demanda el uso de las herramientas adecuadas, accesibles y confiables.
    \item El aprovechamiento de datos es ineficiente cuando los datos se almacenan de manera analógica o manual y no brindan las métricas necesarias para ayudar en la toma de decisiones.
    \item La información no es facilmente accesible ante la falta de una fuente única de verdad en cuanto a los datos almacenados.
    \item El seguimiento y monitoreo de proyectos no logra realizarse en tiempo real y de forma eficiente utilizando medios analógicos.
    \item Realizar un cambio de herramientas implica cierto nivel de complejidad en la adopción de las mismas.
    \item La gestión de proyectos, llevada de manera poco eficiente, también dificulta el manejo presupuestario y su correspondiente seguimiento a nivel contable.
    \item Se busca una manera de registrar, procesar y, especialmente, aprovechar los indicadores de cada proyecto realizado.
    \item Ciertas metodologías como ser la Metodología del Marco Lógico no son traducidas en sistemas de software existentes, a pesar de su importancia, y la FUDEP requiere un sistema personalizado y realizado a medida.
    \item No se cuenta con una estructura de datos definida acorde a las necesidades específicas de los distintos proyectos y programas realizados por la Fundación convocante.
    \item No se tienen alertas técnicas en tiempo real respecto al estado de los proyectos en curso que sean oportunas para los distintos actores de la FUDEP.
    \item La adopción de nuevas herramientas implica un esfuerzo por parte de los responsables de usarlo, sin lo cual no se logra un total beneficio de las mismas.
    \item El tiempo destinado a la ejecución del proyecto es reducido, lo cual implica una elección de tecnologías adecuadas para satisfacer esta necesidad.
    \item Se busca evitar el uso de papel en todas las tareas en las que esto sea posible
\end{itemize}