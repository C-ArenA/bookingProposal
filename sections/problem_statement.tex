\section{Identificación de Problemas y Oportunidades}

De acuerdo a lo expuesto por la Fundación La Paz para el Desarrollo y Participación en los TdR para la implementación de un Sistema Informático de Seguimiento y Monitoreo Institucional,
el análisis realizado y expuesto en este documento y la experiencia técnica de 2IES, se identifican las siguientes problemáticas traducidos en oportunidades 
que fundamentarán el desarrollo de este proyecto y que guiarán la consiguiente propuesta de solución y el plan de trabajo correspondiente:

\begin{itemize}
    \item La implementación de un sistema de gestión moderno, basado en tecnologías de la información, permitirá optimizar los procesos y aumentar la eficiencia en la ejecución de los proyectos y programas de la FUDEP.
    \item La creación de una plataforma digital transparente y accesible fortalecerá la confianza de los financiadores, empleados, comunidad, socios y co partes; y permitirá una mejor comunicación y colaboración.
    \item La transparencia institucional en la era digital demanda el uso de las herramientas adecuadas, accesibles y confiables.
    \item El cambio de técnicas analógicas a digitales en el proceso de recolección y almacenamiento de datos brinda la posibilidad de agilizar y mejorar el aprovechamiento de los mismos otorgando incluso nuevas métricas que ayudan en la toma de decisiones.
    \item La consolidación de una base de datos centralizada y actualizada facilitará el acceso a la información de forma consistente y veraz.
    \item La coordinación entre proyectos realizados en distintos departamentos o regiones puede potenciarse con el uso de plataformas conectadas a la nube o accesibles mediante internet.
    \item Los medios digitales consiguen que el seguimiento y monitoreo de proyectos y programas logren realizarse de forma muy cercana al tiempo real.
    \item La adopción de nuevas herramientas tecnológicas representa una oportunidad para potenciar las capacidades de la institución y mejorar la gestión de los proyectos.
    \item La implementación de un sistema de gestión de proyectos integrado permitirá optimizar el control presupuestario y financiero, asegurando una mayor transparencia y eficiencia.
    \item Se desear desarrollar un sistema de indicadores que permita medir el impacto de los proyectos/programas y tomar decisiones basadas en datos concretos.
    \item La adaptación de la Metodología del Marco Lógico a un sistema de software personalizado permitirá optimizar la planificación, ejecución y evaluación de nuestros proyectos.
    \item En el ámbito de la personalización, un sistema institucional hecho a medida permite recoger necesidades específicas y la generación de estructuras de datos que brinden mayor satisfacción a los usuarios y administradores.
    \item Un sistema de software permite realizar varios tipos de alertas técnicas para gestionar los distintos proyectos realizados.
    \item La adopción de nuevas herramientas implica un esfuerzo por parte de los responsables de usarlo, con lo cual se conseguirá un mayor beneficio de las mismas.
    \item La implementación de una solución tecnológica ágil y escalable es fundamental para cumplir con los plazos establecidos y maximizar el impacto de los proyectos.
    \item La eliminación del papel es una iniciativa que aporta múltiples beneficios y se alinea con las tendencias actuales de sostenibilidad y eficiencia. Es una inversión a largo plazo que se verá reflejada en una mayor productividad, reducción de costos y mejora de la imagen de la organización.
\end{itemize}