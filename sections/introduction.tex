\section{Resumen}

En fecha 27 de Noviembre del año presente se tomaron los requerimientos por parte del cliente para solicitar la propuesta y cotización para la realización de un \textquote{Sistema de Reservas de Hoteles}. 

Dicho sistema, al que en adelante haremos referencia como SRH, tiene como propósito brindar un canal de reservas a nivel nacional que brinde un trato cercano a los dueños de hoteles (anfitriones), que permita responder a particularidades que otras plataformas no poseen, que genere confianza en los usuarios finales (huéspedes) y que además otorgue herramientas necesarias de administración.

De acuerdo a nuestro análisis, que se detalla más adelante, proponemos un MVP (Minimum Viable Product) para rápida prueba de concepto con un enfoque modular y progresivo, siendo 7 los módulos identificados:
1) administración de usuarios,
2) reservas y valoraciones (Pieza central),
3) administración de anfitrión y registro de datos de hoteles,
4) administración general,
5) pagos en línea,
6) facturación electrónica,
7) contabilidad básica,

\textit{Los detalles acerca de las tecnologías a usar, la arquitectura y la infraestructura propuesta se muestran en secciones posteriores.}

Entendiendo que el core del sistema y los demás módulos pueden ser \textbf{implementados de forma independiente y en varios pasos}, los costos del servicio serán los siguientes:

\begin{table}[h!]
    \centering
    \resizebox{\textwidth}{!}{%
    \begin{tabular}{@{}|l|l|@{}}
        \hline
    \rowcolor{jucuxBlue} 
    {\addfontfeature{Color=white} Etapa} & {\addfontfeature{Color=white}Costo}  \\ \midrule
    Core del sistema: Reservas y Administración & Bs. 5200                     \\ \midrule
Módulo de pago 100\% en línea               & Bs. 1300                     \\ \midrule
Módulo de facturación electrónica integrado & Bs. 2000                     \\ \bottomrule
\end{tabular}%
    }
\end{table}

El core incluirá un sistema de pagos por QR con asistencia humana, pero que no requiera el \textbf{gasto en comisiones para terceros}.

El tiempo de desarrollo del core será de 2 meses y medio. Los demás módulos implicarán 1 mes cada uno.
